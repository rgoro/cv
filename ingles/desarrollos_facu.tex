\subparagraph{Para la facultad}
Listo s'olo los que considero de inter'es, dejando afuera las materias, cuyos
proyectos tienen escasa diferencia a\~no a a\~no.
\begin{itemize}
	\item \textbf{Mini procesador RISC} Trabajo Pr'actico para Organizaci'on
		del computador 1.  ``Implementamos'', usando VHDL una ALU de 32 bits y
		un peque\~no procesador uniciclo.

	\item \textbf{Plotter de curvas de F1}  Trabajo pr'actico de M'etodos
		num'ericos.  Consist'ia en representar mediante Splines c'ubicas una
		curva del circu'ito de F'ormula 1 de China, y buscar la trayectoria
		m'as r'apida, con ciertos l'imites en la aceleraci'on.  La
		implementaci'on fue en \texttt{C++}.  Como detalle extra
		implementamos una interfaz gr'afica usando SDL.

	\item \textbf{Tester de rutinas en assembler}  Trabajo pr'actico final para Organizaci'on del
		Computador II.  El sistema deb'ia tomar el c'odigo fuente de una
		funci'on escrita en assembler (x86), una definici'on de casos de test,
		escritos en un lenguaje definido \emph{ad hoc}, y verificar que los
		tests se cumplieran.  Usamos \textbf{Perl, C, C++} y algo de assembler.

	\item \textbf{Buscador en textos} TP para Algoritmos y Estructuras de Datos
		III.  Buscador de palabras sobre una base de datos de archivos de
		texto.  Parte importante del trabajo era desarrollar un \emph{Red-Black
		tree} en \texttt{C++}
	\item \textbf{Tab'u Search} Varias herur'isticas para resolver el problema del 
		viajante de comercio.  La m'as importante de ellas era una implementaci'on del
		algoritmo de Tab'u search (ver $\\$
		\texttt{http://en.wikipedia.org/wiki/Tabu\_search})
	\item \textbf{``Procesos Bar''} Trabajo pr'actico para Ingenier'ia del
		Software 1.  Se trata de un bar con precios din'amicos.  El desarrollo
		fue en \texttt{Java}, con una interfaz en \texttt{JSP}
	\item \textbf{Detector de movimiento} usando la librer'ia OpenCV de Intel
		para la materia Procesamiento Multimedia
	\item \textbf{Kernel} Kernel rudimentario desarrollado en C para la materia
		Programaci'on de Sistemas Operativos.
\end{itemize}

